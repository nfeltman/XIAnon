\documentclass{article}
\usepackage{fullpage}
\usepackage{amsmath}
\usepackage{amssymb}
\usepackage{amsfonts}
\usepackage{stmaryrd}
\usepackage{hyperref}

\usepackage{tikz}
\usepackage{tkz-graph}
\usetikzlibrary{shapes,arrows}

\newcommand{\entrynode}[1]{
  \SetVertexNormal[Shape      = circle,
                   FillColor  = black,
                   LineWidth  = 0pt,
                   MinSize    = 0pt]
  \Vertex[L={\tiny\,}]{#1}
  \SetVertexNormal[Shape      = circle,
                   FillColor  = white,
                   LineWidth  = 2pt]
}

\SetUpEdge[lw         = 1.5pt,
           color      = black,
           labelcolor = white,
           labeltext  = red,
           labelstyle = {sloped,draw,text=blue}]

\tikzset{node distance = 2cm}

\title{Anonymity in XIA}
\author{Nicolas Feltman, David Naylor}
\begin{document}
\maketitle
\section{Introduction}
Big themes: XIA can do anonymity well;  It's about more than technical architeture
\section{Background}
all David
\subsection{XIA}
\subsection{Anonymity}
\section{Approach}
\subsection{Proxies}
Nico
\subsection{Temporary SIDs}
David
\subsection{Principal-Based Control}
David
\section{Comparison}
\subsection{Services vs. Features}
\subsection{Threats vs. ``Meatures''}
\section{Implementation}
We implemented a single proxy and a version of the API.
\subsection{Single Proxy DAG Manipulation}
Nico
\subsection{In-Network Services Issues}
Nico
\subsection{OS Integration and API}
David

\bibliography{FinalWriteup}
\bibliographystyle{plain}
\end{document}