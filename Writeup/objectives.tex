\documentclass{article}
\usepackage{amsmath}
\begin{document}
\section{Hello, David}
These are pretty much just my notes.  I'm writing them in \LaTeX mostly for practice.  This can probably transform into a final document, so feel free to make changes on any level.  I'm not even trying to word this very well.
\section{Desirable Properties}
There are three main properties that a proxy-anonymization system might want to satisfy.
\subsection{Anonymity to Server}
The server cannot determine the identity of the client.
\subsection{Encrypted to Proxy}
The proxy has no way of figuring out the data.  This will of course require that the client get some sort of public key for the server.
\subsection{Server Unawares}
The server is unaware that it is being commuicated with via a proxy.
\section{Scenarios}
In this section, we present scenarios that would warrant the properties above.
\subsection{ Proxy}
\subsection{}
\section{Single Proxy}
In this section we present a single proxy. 
\begin{enumerate}
\item Generate
\end{enumerate}
\section{PDRSA}
\subsection{Algorithm}
Here we present the algorithm for Piecewise-Decrypt RSA.  The main difference now is that there are now two private keys, one of which is kept and the other of which is passed  to the proxy.  I forgot why this distinction is important.  I may have wasted a few hours on this.
\subsubsection{Key generation}
Let $p$, $q$ be large primes like in RSA. Define $n = pq$. Note that $\phi(n) = (p-1)(q-1)$. Now select $e$, $d_2$ that are coprime with $\phi(n)$.  This implies that $ed_2$ is also coprime with $\phi(n)$.  While $e$ can be selected to minimize the cost of encryption, $d_2$ must be selected randomly.  Let $d_2 = (d_1e)^{-1} \mod(p-1)(q-1)$.  

Publish $(e,n)$ as the public key.  $(d_2, n)$ is the semiprivate key.  $(d_1, n)$ is the private key. 
\subsubsection{Encryption}
$c = m^{e} \mod n$
\subsubsection{First Pass Decryption}
$h = c^{d_2} \mod n$
\subsubsection{Final Decryption}
$m = h^{d_1} \mod n$
\subsection{Proof of Correctness}
Pretty similar to the one on wikipedia for RSA.  Fermat's is easier to understand, but Euler's is shorter.
\end{document}